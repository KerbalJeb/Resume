\documentclass{article}
\raggedright

\author{Ben Kellman}
\title{Resume}

\usepackage[explicit]{titlesec}
\usepackage{titling}
\usepackage[margin=0.5in]{geometry}
\usepackage{enumitem}
\usepackage[hidelinks]{hyperref}
\usepackage{qrcode}
\usepackage{ragged2e}
\usepackage[none]{hyphenat}

\renewcommand\familydefault{\sfdefault}
\pagenumbering{gobble}


\titleformat{\section}
{\bfseries\normalsize}
{\MakeUppercase{#1}}
{0em}
{\vspace{-0.75em}}[\rule{\textwidth}{1.5pt}]

\titlespacing\section{0pt}{12pt plus 4pt minus 2pt}{0pt plus 2pt minus 2pt}

\newcommand{\grayhline}{
  \noindent\makebox[\linewidth]{\textcolor{gray}{\rule{0.95\textwidth}{1pt}}}\\
}
%TODO Clean up these commands to make them more useful
\newcommand{\minicolumns}[2]{
  \begin{minipage}[t]{0.2\textwidth}
    \begin{flushright}
      #1
    \end{flushright}
  \end{minipage}
  \hfill
  \begin{minipage}[t]{0.75\textwidth}
    #2
  \end{minipage}
}

\newcommand{\resumeSection}[4]{
      \textbf{#1} \hfill #2\\
      \if\relax\detokenize{#3}\relax
      \else
        #3
      \fi    
      \if\relax\detokenize{#4}\relax
      \else
      \begin{itemize}
        \justifying
        \setlength\itemsep{-0.1em}
        #4
    \end{itemize}
      \fi    

}

\begin{document}

\begin{center}
  \textbf{\Large{\MakeUppercase{Ben Kellman}}}\\
  \underline{\href{mailto: kellman.ben@gmail.com}{kellman.ben@gmail.com}}

\end{center}

\section{experience}

\resumeSection{Center for Aerospace Research (UVic)}{September 2021 - August 2022}
{Participated in the development of ORCASat, a student-built and designed 2u CubeSat, that was launched to provide an optical reference for calibrating ground-based telescopes. Worked on firmware development for the electrical power system and the on-board computer. (See \url{https://www.orcasat.ca/})}
{
  \item Independently developed EPS telemetry collection firmware to sample voltage from 31 different sensors measuring current, voltage and temperature and generate statistics based on each sensor that provided valuable data during flight
  \item Independently developed a C\# (WPF) application to interface with the EPS firmware and generate live plots of the telemetry that was used extensively during the pre-launch testing and integration of the satellite
  \item Developed Python scripts to automated test equipment used to measure battery capacity and recharge efficiency that where used to choose and qualify flight cells
  \item Worked on many small improvements and bug fixes for the FreeRTOS based on-board computer, such as improving the file system performance by 10x
  \item Developed new and fixed existing HIL (hardware in the loop) tests for the satellite using Python and PyTest to improve test coverage and reliability
}

\resumeSection{Schneider Electric}{September 2019 - April 2020}
  {Worked on the firmware test team to develop HIL and manual tests for smart power meters}{
  \item Worked with PyTest to develop automated HIL to test new features, such as sftp file transfers on the smart power meters
  \item Wrote and executed manual tests cases that uncovered bugs related to the power meter's display user interface
  \item Worked closely with both dev and test teams to develop new test cases
}

\section{projects}
\resumeSection{USB Firmware Driver}{}{
    \item Wrote a driver to implement a USB virtual com port
    \item Written in C for a STM32F0 MCU on a custom PCB
    \item Used a logic analyzer to debug the driver
    \item Greatly reduced the RAM and FLASH footprint compared to the vendor
    provided USB stack
}

\grayhline
\resumeSection{Alarm Clock}{}{
    \item Design project to create an Arduino based alarm clock
    \item Worked with a partner to design custom PCB in KiCAD
    \item Remotely collaborated on code and PCB using GitHub
}

\grayhline
\resumeSection{Unity3D Game (C\#)}{}{
    \item Created a basic tile-based spaceship game in Unity3D with two other students
    \item Created a basic UI and JSON based saving/loading system
    \item Optimized preformance to handle a large number of tiles being dynamically destroyed
}

\grayhline
\resumeSection{Magnetic Levitator}{}{
    \item Suspended a neodymium magnet through the use of an electromagnet
    \item Utilized an Arduino and hall effect sensors to provide feedback control(PID)
}



\section{Education}
\resumeSection{University of Victoria}{September 2017 - December 2023}{Bachelor of Engineering in Computer Engineering, GPA:8.8/9.0}{}
\section{Skills}
\textbf{Programming:} C, C++, C\#, Matlab, Python, PyTest, Catch2, CMake, FreeRTOS, Bare Metal, Git, ARM, AVR, UART, I2C, USB, SPI\\
\textbf{Hardware:} Logic Analyzer, Oscilloscope, SMD \& THT Soldering and Rework, Altium, KiCad, Solidworks, DC Load, 3D Printing\\

\end{document}